\section{\module{alsaaudio}}

%\declaremodule{builtin}{alsaaudio}	% standard library, in C
\declaremodule{extension}{alsaaudio}	% not standard, in C

\platform{Linux}

\moduleauthor{Casper Wilstrup}{cwi@aves.dk} % Author of the module code;


\modulesynopsis{ALSA sound support}


The \module{alsaaudio} module defines functions and classes for using
ALSA.

% ---- 3.1. ----
% For each function, use a ``funcdesc'' block.  This has exactly two
% parameters (each parameters is contained in a set of curly braces):
% the first parameter is the function name (this automatically
% generates an index entry); the second parameter is the function's
% argument list.  If there are no arguments, use an empty pair of
% curly braces.  If there is more than one argument, separate the
% arguments with backslash-comma.  Optional parts of the parameter
% list are contained in \optional{...} (this generates a set of square
% brackets around its parameter).  Arguments are automatically set in
% italics in the parameter list.  Each argument should be mentioned at
% least once in the description; each usage (even inside \code{...})
% should be enclosed in \var{...}.

\begin{funcdesc}{mixers}{\optional{cardname}}
List the available mixers. The optional \var{cardname} specifies which
card should be queried (this is only relevant if you have more than one
sound card). Omit to use the default sound card.
\end{funcdesc}

\begin{classdesc}{PCM}{\optional{type}, \optional{mode}, \optional{cardname}}
  This class is used to represent a PCM device (both playback and
  capture devices).
  The arguments are: \\
  \var{type} - can be either PCM_CAPTURE or PCM_PLAYBACK (default). \\
  \var{mode} - can be either PCM_NONBLOCK, PCM_ASYNC, or PCM_NORMAL (the default).\\
  \var{cardname} - specifies which card should be used (this is only
  relevant if you have more than one sound card). Omit to use the
  default sound card
\end{classdesc}

\begin{classdesc}{Mixer}{\optional{control}, \optional{id}, \optional{cardname}}
This class is used to access a specific ALSA mixer.
The arguments are: \\
\var{control} - Name of the chosen mixed (default is Master). \\
\var{id} - id of mixer (default is 0) -- More explanation needed here\\
\var{cardname} specifies which card should be used (this is only relevant 
if you have more than one sound card). Omit to use the default sound card
\end{classdesc}


\begin{excdesc}{ALSAAudioError}
  Exception raised when an operation fails for a ALSA specific reason.
  The exception argument is a string describing the reason of the
  failure.
\end{excdesc}

\subsection{PCM Terminology and Concepts}

In order to use PCM devices it is useful to be familiar with some concepts and
terminology.

\begin{description}
\item[Sample] PCM audio, whether it is input or output, consists at
  the lowest level of a number of single samples. A sample represents
  the sound in a single channel in a brief interval. If more than one
  channel is in use, more than one sample is required for each
  interval to describe the sound. Samples can be of many different
  sizes, ranging from 8 bit to 64 bit presition. The specific format
  of each sample can also vary - they can be big endian byte order,
  little endian byte order, or even floats.

\item[Frame] A frame consists of exactly one sample per channel. If
  there is only one channel (Mono sound) a frame is simply a single
  sample. If the sound is stereo, each frame consists of two samples,
  etc.

\item[Frame size] This is the size in bytes of each frame. This can
  vary a lot: if each sample is 8 bits, and we're handling mono sound,
  the frame size is one byte. Similarly in 6 channel audio with 64 bit
  floating point samples, the frame size is 48 bytes

\item[Rate] PCM sound consists of a flow of sound frames. The sound
  rate controls how often the current frame is replaced. For example,
  a rate of 8000 Hz means that a new frame is played or captured 8000
  times per second.

\item[Data rate] This is the number of bytes, which must be recorded
  or provided per second at a certain frame size and rate.

  8000 Hz mono sound with 8 bit (1 byte) samples has a data rate of
  8000 * 1 * 1 = 8 kb/s

  At the other end of the scale, 96000 Hz, 6 channel sound with 64 bit
  (8 bytes) samples has a data rate of 96000 * 6 * 8 = 4608 kb/s
  (almost 5 Mb sound data per second)

\item[Period] When the hardware processes data this is done in chunks
  of frames. The time interval between each processing (A/D or D/A
  conversion) is known as the period. The size of the period has
  direct implication on the latency of the sound input or output. For
  low-latency the period size should be very small, while low CPU
  resource usage would usually demand larger period sizes. With ALSA,
  the CPU utilization is not impacted much by the period size, since
  the kernel layer buffers multiple periods internally, so each period
  generates an interrupt and a memory copy, but userspace can be
  slower and read or write multiple periods at the same time.

\item[Period size] This is the size of each period in Hz. \emph{Not
    bytes, but Hz!.} In \module{alsaaudio} the period size is set
  directly, and it is therefore important to understand the
  significance of this number. If the period size is configured to for
  example 32, each write should contain exactly 32 frames of sound
  data, and each read will return either 32 frames of data or nothing
  at all.

\end{description}

Once you understand these concepts, you will be ready to use the PCM
API. Read on.

\subsection{PCM Objects}
\label{pcm-objects}

The acronym PCM is short for Pulse Code Modulation and is the method
used in ALSA and many other places to handle playback and capture of
sampled sound data.

PCM objects in \module{alsaaudio} are used to do exactly that, either
play sample based sound or capture sound from some input source
(probably a microphone). The PCM object constructor takes the following
arguments:

\begin{classdesc}{PCM}{\optional{type}, \optional{mode}, \optional{cardname}}

\var{type} - can be either PCM_CAPTURE or PCM_PLAYBACK (default).

\var{mode} - can be either PCM_NONBLOCK, PCM_ASYNC, or PCM_NORMAL (the
default).  In PCM_NONBLOCK mode, calls to read will return immediately
independent of wether there is any actual data to read. Similarly,
write calls will return immediately without actually writing anything
to the playout buffer if the buffer is full.

In the current version of \module{alsaaudio} PCM_ASYNC is useless,
since it relies on a callback procedure, which can't be specified through
this API yet.

\var{cardname} - specifies which card should be used (this is only
relevant if you have more than one sound card). Omit to use the
default sound card

This will construct a PCM object with default settings:

Sample format: PCM_FORMAT_S16_LE \\
Rate: 8000 Hz \\
Channels: 2 \\
Period size: 32 frames \\
\end{classdesc}

PCM objects have the following methods:

\begin{methoddesc}[PCM]{pcmtype}{}
  Returns the type of PCM object. Either PCM_CAPTURE or PCM_PLAYBACK.
\end{methoddesc}

\begin{methoddesc}[PCM]{pcmmode}{}
  Return the mode of the PCM object. One of PCM_NONBLOCK, PCM_ASYNC,
  or PCM_NORMAL
\end{methoddesc}

\begin{methoddesc}[PCM]{cardname}{}
  Return the name of the sound card used by this PCM object.
\end{methoddesc}

\begin{methoddesc}[PCM]{setchannels}{nchannels}
  Used to set the number of capture or playback channels. Common
  values are: 1 = mono, 2 = stereo, and 6 = full 6 channel audio. Few
  sound cards support more than 2 channels
\end{methoddesc}

\begin{methoddesc}[PCM]{setrate}{rate}
  Set the sample rate in Hz for the device. Typical values are 8000
  (poor sound), 16000, 44100 (cd quality), and 96000
\end{methoddesc}

\begin{methoddesc}[PCM]{setformat}{format}
  The sound \var{format} of the device. Sound format controls how the PCM
  device interpret data for playback, and how data is encoded in
  captures.

The following formats are provided by ALSA:
\begin{tableii}{l|l}{Formats}{Format}{Description}
  \lineii{PCM_FORMAT_S8}{Signed 8 bit samples for each channel}
  \lineii{PCM_FORMAT_U8}{Signed 8 bit samples for each channel}
  \lineii{PCM_FORMAT_S16_LE}{Signed 16 bit samples for each channel
    (Little Endian byte order)} 
  \lineii{PCM_FORMAT_S16_BE}{Signed 16
    bit samples for each channel (Big Endian byte order)}
  \lineii{PCM_FORMAT_U16_LE}{Unsigned 16 bit samples for each channel
    (Little Endian byte order)} 
  \lineii{PCM_FORMAT_U16_BE}{Unsigned 16
    bit samples for each channel (Big Endian byte order)}
  \lineii{PCM_FORMAT_S24_LE}{Signed 24 bit samples for each channel
    (Little Endian byte order)} 
  \lineii{PCM_FORMAT_S24_BE}{Signed 24
    bit samples for each channel (Big Endian byte order)}
  \lineii{PCM_FORMAT_U24_LE}{Unsigned 24 bit samples for each channel
    (Little Endian byte order)} 
  \lineii{PCM_FORMAT_U24_BE}{Unsigned 24
    bit samples for each channel (Big Endian byte order)}
  \lineii{PCM_FORMAT_S32_LE}{Signed 32 bit samples for each channel
    (Little Endian byte order)} 
  \lineii{PCM_FORMAT_S32_BE}{Signed 32
    bit samples for each channel (Big Endian byte order)}
  \lineii{PCM_FORMAT_U32_LE}{Unsigned 32 bit samples for each channel
    (Little Endian byte order)} 
  \lineii{PCM_FORMAT_U32_BE}{Unsigned 32
    bit samples for each channel (Big Endian byte order)}
  \lineii{PCM_FORMAT_FLOAT_LE}{32 bit samples encoded as float.
    (Little Endian byte order)} 
  \lineii{PCM_FORMAT_FLOAT_BE}{32 bit
    samples encoded as float (Big Endian byte order)}
  \lineii{PCM_FORMAT_FLOAT64_LE}{64 bit samples encoded as float.
    (Little Endian byte order)} 
  \lineii{PCM_FORMAT_FLOAT64_BE}{64 bit
    samples encoded as float. (Big Endian byte order)}
  \lineii{PCM_FORMAT_MU_LAW}{A logarithmic encoding (used by Sun .au
    files)} 
  \lineii{PCM_FORMAT_A_LAW}{Another logarithmic encoding}
  \lineii{PCM_FORMAT_IMA_ADPCM}{a 4:1 compressed format defined by the
    Interactive Multimedia Association} \lineii{PCM_FORMAT_MPEG}{MPEG
    encoded audio?}  
  \lineii{PCM_FORMAT_GSM}{9600 bits/s constant rate encoding for speech}
\end{tableii}

\end{methoddesc}

\begin{methoddesc}[PCM]{setperiodsize}{period}
  Sets the actual period size in frames. Each write should consist of
  exactly this number of frames, and each read will return this number
  of frames (unless the device is in PCM_NONBLOCK mode, in which case
  it may return nothing at all)
\end{methoddesc}

\begin{methoddesc}[PCM]{read}{}
  In PCM_NORMAL mode, this function blocks until a full period is
  available, and then returns a tuple (length,data) where
  \emph{length} is the number of frames of captured data, and
  \emph{data} is the captured sound frames as a string. The length of
  the returned data will be periodsize*framesize bytes.

  In PCM_NONBLOCK mode, the call will not block, but will return
  \code{(0,'')} if no new period has become available since the last
  call to read.
\end{methoddesc}

\begin{methoddesc}[PCM]{write}{data}
  Writes (plays) the sound in data. The length of data \emph{must} be
  a multiple of the frame size, and \emph{should} be exactly the size
  of a period. If less than 'period size' frames are provided, the
  actual playout will not happen until more data is written.

  If the device is not in PCM_NONBLOCK mode, this call will block if
  the kernel buffer is full, and until enough sound has been played to
  allow the sound data to be buffered. The call always returns the
  size of the data provided

  In PCM_NONBLOCK mode, the call will return immediately, with a
  return value of zero, if the buffer is full. In this case, the data
  should be written at a later time.
\end{methoddesc}

\begin{methoddesc}[PCM]{pause}{\optional{enable=1}}
  If \var{enable} is 1, playback or capture is paused. If \var{enable} is 0, 
  playback/capture is resumed.
\end{methoddesc}

\strong{A few hints on using PCM devices for playback}

The most common reason for problems with playback of PCM audio, is
that the people don't properly understand that writes to PCM devices
must match \emph{exactly} the data rate of the device.

If too little data is written to the device, it will underrun, and
ugly clicking sounds will occur. Conversely, of too much data is
written to the device, the write function will either block
(PCM_NORMAL mode) or return zero (PCM_NONBLOCK mode).

If your program does nothing, but play sound, the easiest way is to
put the device in PCM_NORMAL mode, and just write as much data to the
device as possible. This strategy can also be achieved by using a
separate thread with the sole task of playing out sound.

In GUI programs, however, it may be a better strategy to setup the
device, preload the buffer with a few periods by calling write a
couple of times, and then use some timer method to write one period
size of data to the device every period. The purpose of the preloading
is to avoid underrun clicks if the used timer doesn't expire exactly
on time.

Also note, that most timer APIs that you can find for Python will
cummulate time delays: If you set the timer to expire after 1/10'th of
a second, the actual timeout will happen slightly later, which will
accumulate to quite a lot after a few seconds. Hint: use time.time()
to check how much time has really passed, and add extra writes as
nessecary.

\subsection{Mixer Objects}
\label{mixer-objects}

Mixer objects provides access to the ALSA mixer API.

\begin{classdesc}{Mixer}{\optional{control}, \optional{id}, 
    \optional{cardname}}
  \var{control} - specifies which control to manipulate using this
  mixer object. The list of available controls can be found with the
  \module{alsaaudio}.\function{mixers} function.  The default value is
  'Master' - other common controls include 'Master Mono', 'PCM',
  'Line', etc.

  \var{id} - the id of the mixer control. Default is 0

  \var{cardname} - specifies which card should be used (this is only
  relevant if you have more than one sound card). Omit to use the
  default sound card
\end{classdesc}

Mixer objects have the following methods:

\begin{methoddesc}[Mixer]{cardname}{}
  Return the name of the sound card used by this Mixer object
\end{methoddesc}

\begin{methoddesc}[Mixer]{mixer}{}
  Return the name of the specific mixer controlled by this object, For
  example 'Master' or 'PCM'
\end{methoddesc}

\begin{methoddesc}[Mixer]{mixerid}{}
  Return the ID of the ALSA mixer controlled by this object.
\end{methoddesc}

\begin{methoddesc}[Mixer]{switchcap}{}
  Returns a list of the switches which are defined by this specific
  mixer. Possible values in this list are:

\begin{tableii}{l|l}{Switches}{Switch}{Description}
  \lineii{'Mute'}{This mixer can be muted} 
  \lineii{'Joined Mute'}{This mixer can mute all channels at the same time} 
  \lineii{'Playback Mute'}{This mixer can mute the playback output} 
  \lineii{'Joined Playback Mute'}
  {Mute playback for all channels at the same time}
  \lineii{'Capture Mute'}{Mute sound capture} 
  \lineii{'Joined Capture Mute'}{Mute sound capture for all channels at a time}
  \lineii{'Capture Exclusive'}{Not quite sure what this is}
\end{tableii}

To manipulate these swithes use the \method{setrec} or
\method{setmute} methods
\end{methoddesc}

\begin{methoddesc}[Mixer]{volumecap}{}
  Returns a list of the volume control capabilities of this mixer.
  Possible values in the list are:

\begin{tableii}{l|l}{Volume Capabilities}{Capability}{Description}
  \lineii{'Volume'}{This mixer can control volume}
  \lineii{'Joined Volume'}{This mixer can control volume for all channels at 
    the same time}
  \lineii{'Playback Volume'}{This mixer can manipulate the playback volume}
  \lineii{'Joined Playback Volume'}{Manipulate playback volumne for all 
    channels at the same time}
  \lineii{'Capture Volume'}{Manipulate sound capture volume}
  \lineii{'Joined Capture Volume'}{Manipulate sound capture volume for all 
    channels at a time}
\end{tableii}

\end{methoddesc}

\begin{methoddesc}[Mixer]{getrange}{\optional{direction}}
  Return the volume range of the ALSA mixer controlled by this object.

  The optional \var{direction} argument can be either 'playback' or
  'capture', which is relevant if the mixer can control both playback
  and capture volume.  The default value is 'playback' if the mixer
  has this capability, otherwise 'capture'

\end{methoddesc}

\begin{methoddesc}[Mixer]{getvolume}{\optional{direction}}
  Returns a list with the current volume settings for each channel.
  The list elements are integer percentages.

  The optional \var{direction} argument can be either 'playback' or
  'capture', which is relevant if the mixer can control both playback
  and capture volume. The default value is 'playback' if the mixer has
  this capability, otherwise 'capture'

\end{methoddesc}

\begin{methoddesc}[Mixer]{getmute}{}
  Return a list indicating the current mute setting for each channel.
  0 means not muted, 1 means muted.

  This method will fail if the mixer has no playback switch
  capabilities.
\end{methoddesc}

\begin{methoddesc}[Mixer]{getrec}{}
  Return a list indicating the current record mute setting for each
  channel. 0 means not recording, 1 means recording.

  This method will fail if the mixer has no capture switch
  capabilities.
\end{methoddesc}

\begin{methoddesc}[Mixer]{setvolume}{volume,\optional{channel},
    \optional{direction}}

  Change the current volume settings for this mixer. The \var{volume}
  argument controls the new volume setting as an integer percentage.

  If the optional argument \var{channel} is present, the volume is set
  only for this channel. This assumes that the mixer can control the
  volume for the channels independently.

  The optional \var{direction} argument can be either 'playback' or
  'capture' is relevant if the mixer has independent playback and
  capture volume capabilities, and controls which of the volumes if
  changed. The default is 'playback' if the mixer has this capability,
  otherwise 'capture'.
\end{methoddesc}

\begin{methoddesc}[Mixer]{setmute}{mute, \optional{channel}}
  Sets the mute flag to a new value. The \var{mute} argument is either
  0 for not muted, or 1 for muted.

  The optional \var{channel} argument controls which channel is muted.
  The default is to set the mute flag for all channels.

  This method will fail if the mixer has no playback mute capabilities
\end{methoddesc}

\begin{methoddesc}[Mixer]{setrec}{capture,\optional{channel}}
  Sets the capture mute flag to a new value. The \var{capture}
  argument is either 0 for no capture, or 1 for capture.

  The optional \var{channel} argument controls which channel is
  changed. The default is to set the capture flag for all channels.

  This method will fail if the mixer has no capture switch
  capabilities.
\end{methoddesc}


\textbf{A Note on the ALSA Mixer API}

The ALSA mixer API is extremely complicated - and hardly documented at
all. \module{alsaaudio} implements a much simplified way to access
this API. In designing the API I've had to make some choices which may
limit what can and cannot be controlled through the API. However, If I
had chosen to implement the full API, I would have reexposed the
horrible complexity/documentation ratio of the underlying API.  At
least the \module{alsaaudio} API is easy to understand and use.

If my design choises prevents you from doing something that the
underlying API would have allowed, please let me know, so I can
incorporate these need into future versions.

If the current state of affairs annoy you, the best you can do is to
write a HOWTO on the API and make this available on the net. Until
somebody does this, the availability of ALSA mixer capable devices
will stay quite limited.

Unfortunately, I'm not able to create such a HOWTO myself, since I
only understand half of the API, and that which I do understand has
come from a painful trial and error process.



% ==== 4. ====
\subsection{ALSA Examples \label{pcm-example}}

For now, the only examples available are the 'playbacktest.py' and the
'recordtest.py' programs included.  This will change in a future
version.
